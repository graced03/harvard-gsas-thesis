\begin{appendices}

\chapter{Appendix to Chapter \ref{ch:1}}\label{cha:append-chapt-refch:1}

\section{Auxiliary Lemmata}
Fundamental identity
\begin{equation}
  \label{eq:A}
  e^{i\pi}=-1.
\end{equation}
Equivalence relation
\begin{equation}
  \label{eq:B}
  A=B.
\end{equation}

\section{Proofs}

\chapter{Appendix to Chapter \ref{ch:3}}

\section{Proofs}

\section{Supplementary Tables and Figures}
\begin{longtable}{cc}
% Caption to appear on the first page and in the list of tables
\caption[Optional Short caption (used in list of tables)]{A long table} \label{grid_mlmmh} \\

Heading that appears & on first page only\\
\hline
\endfirsthead
\caption[]{(continued)}\\ % Caption to appear on subsequent pages. GSAS requires
                          % that if a table spills over multiple pages, it
                          % contains the (continued) label

Heading that appears & on all pages\\
\hline
\endhead
% optional footer
\hline \multicolumn{2}{r}{{Continued on next page}} \\
\endfoot
% footer to appear on tha last page (empty)
\endlastfoot

Contrary to popular & belief, Lorem Ipsum \\ is & not \\ simply &
random \\ text &. It \\ has & roots \\ in & a \\ piece & of \\ classical & Latin
\\ literature & from \\ 45 & BC \\, making & it \\ over & 2000 \\ years old. Richard & Mc \\Clintock &, a \\ Latin & professor \\ at & Hampden \\-Sydney
& College \\ in & Virginia \\, looked & up \\ one & of \\ the & more
\\obscure & Latin \\ words &, consectetur \\, from & a \\ Lorem & Ipsum \\
passage &, and \\ going & through \\ the & cites \\ of & the word in \\
classical & literature , discovered the \\ undoubtable & source. Lorem Ipsum
\\ comes & from \\ sections & 1 \\.10 &.32 \\ and & 1 \\.10 &.33 \\ of
&"de \\ Finibus & Bonorum \\ et & Malorum \\" (The & Extremes \\ of & Good \\
and & Evil \\) by & Cicero \\, written & in \\ 45 & BC \\. This & book \\ is & a
\\ treatise & on \\ the & theory \\ of & ethics \\, very & popular \\ during
&the \\ Renaissance &. The \\ first & line \\ of & Lorem \\ Ipsum, "Lorem ipsum
& dolor \\ sit & amet \\..", comes from a & line \\ in & section
1.10.32.\\
\end{longtable}


\begin{sidewaysfigure}
  \centering Supplementary figures and tables should be placed in the appendix,
  not at the end of a chapter. To rotate big tables and figures $90^{\circ}$,
  use the rotating package and the sidewaystable and sidewaysfigure
  environments. This ensures that the figure and caption get rotated, but the
  page number stays at the bottom of the page.
  \caption{Supplementary Figure}
  \label{fig:figuresup1}
\end{sidewaysfigure}

\begin{figure}[ht]
  \centering
  This is another supplementary figure.
  \caption{Another Figure}
  \label{fig:figuresup3}
\end{figure}


\end{appendices}
